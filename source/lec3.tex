% lec3.tex 
\lecture{3}{08:14 AM Thu, Oct 09 2025}{} 
\begin{proposition}[]
  \begin{itemize}
    \item 
$(\frac{G}{H}) _{d}$  and 
$\left( \frac{G}{H} \right) _{d} $  are equipotent.
\item  
and for all $x \in   G: \quad xH $  and 
$Hx $ are equipotent.
  \end{itemize}
\end{proposition}
\begin{proof}
\[
\begin{array}{cccc}
      f : &  (\frac{G}{H}) )_{g}  & \longrightarrow & 
      \left( \frac{G}{H} \right)_{d}\\

           &  xH  & \longmapsto     & Hx^{-1} \\ 
\end{array}
\]
we have for all $x,y \in  G$: 
\[
xH = yH \implies 
Hx^{-1} = Hy^{-1} \implies 
f(xH) = f(yH) 
\]
so $f $ is well defined.
\[
f(xH) = f(yH)  \iff 
Hx^{-1} = Hy^{-1} \implies 
xH = yH \implies \text{ f injective }  
\]
Let $Hy \in  \left( \frac{G}{H} \right) _{d} $ then:
\[
f(y^{-1}H)  = Hy \text{ with }  
y^{-1}H \in   
\left( \frac{G}{H} \right) _{g}
\]
hence $f $ is surjective, thus 
$f $ is bijective. Let 
\[
\begin{array}{cccc}
      f : &  xH  & \longrightarrow & Hx \\

           &  xh  & \longmapsto     & hx \\ 
\end{array}
\]
we have that $f $ is bijective.
\end{proof}
\begin{theorem}[Lagrange Theorem]
Let $G $ be a finite group and let 
$H \leq G $, then $\left| H \right|  /  \left| G \right|   $ 
\end{theorem}
\begin{proof}
Let $H \leq G $. then $\left( \frac{G}{H} \right) _{g} $ is a partition 
of $G $, Let $\left( \frac{G}{H} \right) {g} = \left\{ x_1 H, \hdots , x_n H \right\} $ 
for some $n \in  \NN $, so $G = \bigcup_{i=1}^{n} x_i H $. Since 
$\left( \frac{G}{H} \right) _{g} $ is a partition of $G $, then: 
\[
\left| G \right|   = 
\left| \bigcup_{}^{} x_{i}H \right|  = 
\sum_{i=1}^{n} \left| x_{i} H \right|  
\]
we have $H $ and $x_{i} H $ are equipotent for all $i \in   \left\{ 1, \hdots , n \right\} $, 
take 
\[
\begin{array}{cccc}
      f : &  H  & \longrightarrow &  x_{i} H\\

           &   h& \longmapsto     &  x_{i} h\\ 
\end{array}
\]
so 
\[
\left| G \right|   = \sum_{i=1}^{n} \left| H \right|  = n \left| H \right|  
\]
hence $\left| H \right|  \ \left| G \right|   $ 
\end{proof}
\noindent
\textcolor{purple}{
\underline{
\ding{43}Notation:
}\\
}
$n = \left[ G:H \right] $ called the index of $H $ in $G $. if $\left| G \right|  <  +\infty  $,
then \[
\left[ G:H \right] = \frac{\left| G \right|  }{\left| H \right|  }
\]
For $G = \left( \mathbb{Z}, + \right)  $ and $H = n \mathbb{Z}$ where $n \in  \NN $, we have: 
\[
\left[ G:H \right] = \left| \left( \frac{G}{H} \right)  \right|  = n
\]
\begin{corollary}[]
Every finite group $G $ of prime order is cyclic spanned any element $x \in   G \backslash \left\{ e \right\} $ 
\end{corollary}
\begin{proof}
Let $x \in  G \backslash  \left\{  e \right\} $. By lagrange theorem we have 
$\left| \left\langle x \right\rangle  \right|  \backslash  \left| G \right|  = p $. 
so $\left\langle x \right\rangle = \left\{ e \right\} $ or $G = \left\langle x \right\rangle  $. 
since $ x \neq e $, then $\left\langle x \right\rangle \neq \left\{ e \right\} $. So $G = \left\langle x \right\rangle  $ 
\end{proof}
\noindent
\textcolor{red}{
\underline{
\ding{46}Remark:
}
}\\
The reciprical of this result is not true. we can have a cyclic group with a non prime order, 
\[
\frac{\mathbb{Z}}{6 \mathbb{Z}} = 
\left\langle \overline{1} \right\rangle =
\left\langle \overline{5} \right\rangle 
\]
is cyclic and $\left| \frac{\mathbb{Z}}{6 \mathbb{Z}} \right|  = 6 $ not prime.
\subsection{Normal Subgroups}
\begin{definition}[]
Let $H \leq G $. we say that $H $ is a normal group of $G $ if: 
\[
\forall x \in   G: \quad xH = Hx 
\quad \quad \text{(denoted $H \lhd G $)}  
\]
\end{definition}
\noindent
\textcolor{red}{
\underline{
\ding{46}Remark:
}
}\\
If $G $ is abelian, then any subgroup of $G $ is normal
\begin{proposition}[]
Let $H \leq G $. then the following statments are equivallent:
\begin{itemize}
  \item[\ding{172}] $\forall x \in   G: \quad x H x^{-1} = H $ 
  \item[\ding{173}] $\forall x \in  G: \quad x H x^{-1} \subset H $ 
\end{itemize}
\end{proposition}
\begin{proof}
Exercise
\end{proof}
% \section{Quotient Group}
Let $G $ be a group and $(H \lhd  G)$. then:
\[
\left( \frac{G}{H} \right) _{g} = \left( \frac{G}{H} \right) _{d} \overset{\text{ def }  }{=} 
\frac{G}{H}
\]
we equip $\frac{G}{H} $ by the binary operation, defined:
\[
\forall x,y \in   G: \quad (xH) \cdot  (yH) = \overline{x}\cdot \overline{y}= \overline{xy}=
\left( xy \right) \cdot H
\]
\begin{theorem}[]
$\left( \frac{G}{H}, \cdot  \right)  $ is a group called the quotient group of $G $ by 
$H $ 
\end{theorem}
\begin{proof}
  \begin{itemize}
    \item[\ding{49}] Associativity
    \item[\ding{49}] The neutral element of $\left( \frac{G}{H} \right)$ is $H = \overline{e} = e H $ 
    \item[\ding{49}] for all $\overline{x}\in   \frac{G}{H}: \quad \overline{x}^{-1}= \overline{x^{-1}}= 
      \left( x H \right) ^{-1}$ 
  \end{itemize}
\end{proof}
\section{Group Morphism}
\begin{definition}[]
Let $(G, \cdot ) $ and $(G', \mathcal{T} )  $ be two groups. A map
$ f : G \longrightarrow G' $ is a group morphism if for all $x,y \in  G $ we have:
\[
f(x \cdot y) = f(x) \mathcal{T} f(y) 
\]
\end{definition}
\begin{proposition}[]
Let $ f : G \longrightarrow G' $ a Group morphism, and $e, e' $ are
the neutral elements of $G $ and $G' $ resp. Then:
\begin{itemize}
  \item[\ding{172}] $f(e)  = e' $ 
  \item[\ding{173}] $\forall x \in   G: \quad f(x^{-1}) = \left[ f(x)  \right]^{-1}  $ 
  \item[\ding{174}] $\forall x \in G, \quad \forall n \in  \mathbb{Z}: \quad 
    f(x^n ) = \left[ f(x)  \right]^n $ 
\end{itemize}
\end{proposition}
\begin{proof}
  \begin{enumerate}
    \item[\fbox{1}]
  \ding{50} $x = y = e \implies  f(e)  = f(e)  f(e) $ 
\item[\fbox{2}] \ding{50}
      $y = x^{-1} \implies e' = f(x) f(x^{-1})  $ 
    \item[\fbox{3}] \ding{50} use induction
  \end{enumerate}
\end{proof}
\begin{example}
Let $H \lhd G $, then: 
\[
\begin{array}{cccc}
      i : &  H  & \longrightarrow & G \\

           &  h  & \longmapsto     & h \\ 
\end{array}
\]
and \[
\begin{array}{cccc}
      s : &  G  & \longrightarrow & \frac{G}{H} \\

           &  x  & \longmapsto     & \overline{x} = 
           xH\\ 
\end{array}
\]
are group morphism called injective and surjective, resp.
\end{example}
\subsection{Kernel-Image}
Let $ f : G \longrightarrow G'$, The kernel of $f $ 
is: 
\[
  \fbox{ 
    $
  \text{Ker}(f) =
  \left\{ x \in   G: \quad 
  f(x)  = e' \right\} = 
  f^{-1}(\left\{ e' \right\}) 
  $
  }
\]
The image of $f $ is: 
\[
  \fbox{ 
    $
  \text{Im}(f)   = 
  \left\{ f(x): \quad x \in  G \right\} = f(G) 
  $
  }
\]
\begin{proposition}[]
Let $ f : G \longrightarrow G' $ be a 
Group morphish, then:
\begin{itemize}
  \item [\ding{172}] If $H \leq G $, then
    $f(H)  \leq G' $ 
  \item [\ding{173}] $H ' \leq G ' $, then 
    $f^{-1}(H')  \leq  G $ 
\end{itemize}
so 
\[
\text{Im}  (f) = f(G)  \leq  G' 
\]
and 
\[
f^{-1}( \left\{ e' \right\})  = 
\text{Ker}  (f)  \leq G
\]
\end{proposition}
\begin{proposition}[]
Let $ f : G \longrightarrow G' $ be a group
morphism, then: 
\begin{itemize}
  \item [\ding{172}] $f $ injective $\iff  $ 
    $\text{Ker}  (f)  = \left\{ e \right\} $ 
  \item [\ding{173}] $f $ surjective $\iff  $ 
    $\text{Im}  (f) = G'  $ 
\end{itemize}
\end{proposition}
\begin{proof}
  Exercise \tt (you know it won't happen) \ding{44} \normalfont
\end{proof}
\begin{proposition}[]
Let $H \leq G $, then: 
\[
H \lhd G \iff \exists G' \text{ a group and GM $ f : G \longrightarrow G'$ such that: $H = \text{Ker}  (f)  $ }  
\]
\end{proposition}
\begin{proof}
  \[
    ( \impliedby  ) 
  \]
  
  suppose that $\exists G' $ a group, and $ f : G \longrightarrow G' $ a group morphism such that $H = \text{Ker}  (f)$ 
Let $x \in  G $, and let $y = x gx^{-1} \in  x \text{Ker}  (f) x^{-1} $ , hence: 
\begin{align*}
  f(y)  
  &= 
  f(x) f(g) (x^{-1})  \\
  &= 
  f(x) f(x^{-1})  \\
  &= e' 
\end{align*}
so $y \in   \text{Ker}  (f)  $. look at the hand below
\[
  \text{\ding{43} }  x \text{Ker} (f) x^{-1} \subset 
  \text{Ker}(f) 
\]
Therefore $H = \text{Ker}  (f) \lhd G $.
\[
\left( \implies  \right) 
\]
Let $H \lhd  G$, and let \[
\begin{array}{cccc}
      f : &  G  & \longrightarrow & \frac{G}{H} \\

           &  x  & \longmapsto     & \overline{x} =
           x H\\ 
\end{array}
\]
we have $f $ is a group morphism, and $\text{Ker}  (f) = H$ 
\end{proof}
\begin{theorem}[First Theorem of Isomorphism]
  Let $ f : G \longrightarrow G' $ be a group morphism,
  then $\frac{G}{\text{Ker}  (f) } \sim \text{Im}  (f) $ 
\end{theorem}
\begin{proof}
  \ding{45} Let \[
\begin{array}{cccc}
      \overset{\sim}{f}  : &  \frac{G}{\text{Ker}  (f) }  & \longrightarrow & \text{Im}  (f)  \\

           &    \overline{x}& \longmapsto     & 
           \overset{\sim}{f} ( \overline{x}) = 
           f(x) \\ 
\end{array}
\]
Let $\overline{x}, \overline{y} \in  \frac{G}{\text{Ker}  (f) } $ such that $\overline{x} = \overline{y}$, then 
$x \text{Ker}  (f) = y \text{Ker}  (f)$, therefore 
$x^{-1}y \in  \text{Ker}  (f) $, hence we can deduce
\begin{align*}
  f(x^{-1}y)  = e &\iff f(x) = f(y)  \\
                  &\iff \overset{\sim}{f} (\overline{x})  = 
  \overset{\sim}{f} (\overline{y}) 
\end{align*}
so $f $ is well defined and injective. 
$f $ is surjective by construction
\end{proof}
%end of file
