% lec1.tex 
\chapter{Introduction}
\lecture{1}{08:10 AM Thu, Sep 25 2025}{General Overview} 
Let $E \neq \emptyset  $ a set. \\
A binary operation $\cdot  $ on $E $ is any map from 
$E \times E  $ into $E $,
\[
\begin{array}{cccc}
  ( \cdot )  : &  E \times E   & \longrightarrow & E \\

           &  (x,y)   & \longmapsto     & x \cdot y \\ 
\end{array}
\]
Let $A \subset E $, we say $A $ is a stable by $(\cdot)  $if $(\cdot)$ is also a Binary Operation 
on $A$,
\[
\begin{array}{cccc}
  (\cdot _{A}) : &  A \times A   & \longrightarrow & A \\

           &  (x, y)   & \longmapsto     &  x \cdot _{A} y = x \cdot y\\ 
\end{array}
\]

\begin{definition}[Group]
Let $G \neq \emptyset  $ a set with a Binary Operation $(*)$, we say that 
$G $ is a group if : 
\begin{enumerate}
\item $(*)$ is associative, if : 
  \[
  \forall x,y,z \in  G: \quad (x * y ) * z  = x * (y * z) 
  \]
  \item $(*)  $   admits a netural elements  if :
    \[
      \exists e \in  G, \forall x \in G: \quad 
      x * e = e * x = x
    \]
    \item \[
    \forall x \in  G, \exists x' \in G: \quad  x * x' = 
    x' * x = e
    \]
\end{enumerate}
if $(*)$ is commutative i.e.:
\[
\forall x,y \in  G: \quad x * y = y * x
\]
then $G $ is called an Abelian Group. \\
\end{definition}
\underline{\emph{Notation:}} We denote $(*) $ by $(\cdot)$  if its multiplicative,
and $(+) $ if its additive.
\begin{proposition}
  Let $(G, \cdot )  $ be a group. then:
  \begin{enumerate}
  \item The Neutral Element is uniuqe.
    \item The inverse is unique
      \item \[
      \forall x, y \in  G: \quad (x \cdot  y)^{-1} = y^{-1} \cdot x^{-1}
      \]
      \item 
        \[
        \forall x,y,z \in  G: \quad 
        \begin{cases}
        xy = xz \\
        yx = zx
        \end{cases} 
        \implies 
        \begin{cases}
          y = z \\
          y = z
        \end{cases}
        \]
  \end{enumerate}
\end{proposition}
\begin{proof}
\begin{enumerate}
\item Let $e_1, e_2 \in G $ be a Neutral Element, then: 
  \[
    e_1 = e_1 \cdot e_2 = e_2
  \]
\item 
  let $x \in  G $ and $x_1, x_2 \in  G $ be its inverses, then:
  \[
  x_1 = x_1 \cdot  e = x_1 \cdot  (x \cdot  x_2)  = (x_1 \cdot  x) \cdot  x_2 = e \cdot  x_2 = x_2 
  \]
 \item Let $x, y' \in G $. then:
   \begin{align*}
     (x \cdot y)  \cdot (x \cdot  y) ^{-1} = e & \implies 
     y \cdot (x \cdot y) ^{-1} = x^{-1} \\
                                               & \implies 
                                               (x \cdot  y) ^{-1} = y^{-1} \cdot x^{-1}
   \end{align*}
\end{enumerate}
\end{proof}
\exercise Let $(G, \cdot )  $ be a group and $x_1, x_2, \hdots , x_{n} \in G $. then: 
\begin{itemize}
\item  $(x_1 \cdot \cdot  \cdot x_{n}) ^{-1} = x_{n}^{-1} \cdot \cdot \cdot x_1^{-1}$
\item  $(x_1^{-1}) ^{-1} = x_1$
\end{itemize}

\begin{definition}[]
Let $(G, \cdot )  $  be a group, $n \in \mathbb{Z} $ and $x \in  G $, define 
\[
x^n = \begin{cases}
x \cdot x \cdot \cdot \cdot x 
\\
e\\
x^{-1} \cdot  x^{-1} \cdot \cdot \cdot x^{-1}
\end{cases}
\implies 
\begin{cases}
\text{ if } n \geq 1 \\
\text{ if } n = 0 \\
\text{ if } n \leq -1
\end{cases}
\]
\end{definition}
\begin{example}
\begin{enumerate}
\item $(Z, +), (\mathbb{Q}^{*}, \cdot ), (\RR , +), (\CC^{*}, \cdot)$ 
 \item The set $\mathcal{F} (\RR , \RR )  $  with addition of maps is an Abelian Group, with the null map
   as Neutral Element
  \item The set $S_{n} $ of all bijection of $\left\{ 1, \hdots , n \right\} $  with composition 
    of maps is a group 
\end{enumerate}
\end{example}
\begin{definition}[Sub Group]
  Let $(G, \cdot )$ be a group and $H \subset G$ we say that $H $ is a Subgroup of $G $  if 
  $(H, \cdot )  $ is a gorup
\end{definition}

\begin{proposition}
  
  Let $(G, \cdot )  $  a group and $H \subset G $. then $H $ is a Subgroup of $G $ if and only if:
  \begin{enumerate}
  \item $H \neq \emptyset  $ 
    \item $ \forall x,y \in  H: \quad x \cdot  y \in  H $  
      \item $\forall x \in  H: \quad x^{-1} \in  H $  
  \end{enumerate}
\end{proposition}
\underline{\emph{Remark}}: The conditions $(2)$ and $(3)$ are equivalent to:
\[
\forall x,y \in H: \quad x^{-1} \cdot  y \in  H 
\]
\begin{proof}
\[
\forall x,y \in H: \quad x^{-1}\cdot y \in  H \implies 
\begin{cases}
\forall x,y \in  H: \quad x \cdot  y \in  H \\
\forall x \in  H: \quad x^{-1} \in  H
\end{cases}
\]
\end{proof}
\underline{\emph{Notation}}: if $H $ is a Subgroup of $G $, 
we denote 
\[
H \leq G
\]
if $H \leq  G$ with $H \neq G $, we call $H $ a proper Subgroup of $G $ and we write $H < G$   \\
\exercise
Let $(G, \cdot )  $ be a group, then the set:
\[
Z(G) = \left\{ x \in  G: \quad  gx = xg , \forall g \in  G \right\} 
\]
\begin{enumerate}
  \item Prove that $Z(G) = G\iff G \text{ is an abelian group.} $ 
\end{enumerate}
\begin{proof}
  \begin{enumerate}
  \item 
\[
( \implies ) 
\]
Suppose that $G $ is an Abelian Group. \\
Let $x \in  G $ and let $g \in G $, since $G $ is an Abelian group, then $gx = xg $. then $x \in  Z(G)$,
then $Z(G) = G$
\[
 ( \impliedby  ) 
\]
Suppose that $Z(G) = G$, let $x,y \in  G$. then $x,y \in Z(G)$. so $\forall g \in G$ :
\[
\begin{cases}
xg = gx \\
yg = gy
\end{cases}
\]
so for $g = y$, we get $xy = yx $ so $G $ is an abelian group 
\item 
  Let $G \leq (\mathbb{Z}, +)  $.  
  \begin{itemize}
    \item       if $G = \left\{ 0 \right\} $. then $G = 0 \mathbb{Z} $. 
      \item
  if $G \neq \left\{ 0 \right\} $, then $\exists m \in G $ with $m \neq 0$, without loss of generality.
  suppose $m > 0$, so $G \cap  \NN \neq  \emptyset $, so $n = \min G \cap \mathbb{N} $ , let $x \in  n \mathbb{Z} $.
  then $x = kn, k \in  \mathbb{Z}$, so $x \in  G $. hence $n \mathbb{Z} \subset G $. \\
  Let $x \in  G $, so $\exists q,r \in  \mathbb{Z}$, $0 \leq r \leq  n-1$ such that 
  $x = qn + r $. so $r = x - qn \in  G $, if $r \neq  0 $ then : 
  \[
  \begin{cases}
  r < n \\
  r = G \cap \NN
  \end{cases} \implies 
  \begin{cases}
  r < n \\
  n = \min  G \cap \NN \leq r, \quad \text{ is a contradiction } 
  \end{cases}
  \]
  so $x = qn \in  n \mathbb{Z} $, so $G \subset n \mathbb{Z}$ 
  \end{itemize}
  \end{enumerate}
\end{proof}

\begin{proposition}
  Let $H, K \leq G $, with $G $ is a group. then: 
  \[
  H \cap K \leq G
  \]
\end{proposition}
  \begin{proof}
    Since $e \in H $ and $e \in  K$, then $e \in  H \cap K \neq  \emptyset $. \\
    Let $x,y \in  H \cap K $, then : 
    \[
    \begin{cases}
    x,y \in  H \\
    x,y \in  K
    \end{cases} \implies 
    \begin{cases}
    x^{-1}, y \in  H \\
    x^{-1}, y \in  K \\
    \end{cases}
    \implies  
    \begin{cases}
    x^{-1} \cdot y \in H \\
    x^{-1} \cdot y \in K
    \end{cases}
    \implies x^{-1} \cdot y \in H \cap K
    \]
  \end{proof}
 \begin{proposition}
   Let $\left\{ H_{i} \right\}_{i \in  I} $ be a family of Subgroup of a group $G $, then : 
   \[
     \bigcap_{i \in  I}^{} H_{i} \leq G
   \]
 \end{proposition}
\underline{\emph{Remark}}: $H \cup K $ is not always a Subgroup of $G $.
 \begin{proposition}
   Let $H, K \leq G$, Then $H \cup K \leq G \iff  \begin{cases}
   H \subset K \\
   K \subset H
   \end{cases}$  
 \end{proposition}
 
 

% end of file
